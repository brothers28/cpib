%%%%%%%%%%%%%%%%%%%%%%%%%%%%%%%%%%%%%%%%%
% Nat64 support for IML
% Module cpib
%
% This template was downloaded from:
% http://www.LaTeXTemplates.com
%
% Authors:
% Marco Romanutti
% Benjamin Meyer
%
% License:
% CC BY-NC-SA 3.0 (http://creativecommons.org/licenses/by-nc-sa/3.0/)
%
%%%%%%%%%%%%%%%%%%%%%%%%%%%%%%%%%%%%%%%%%

%----------------------------------------------------------------------------------------
%	PACKAGES AND OTHER DOCUMENT CONFIGURATIONS
%----------------------------------------------------------------------------------------

\documentclass[10pt, a4paper, twocolumn]{article} % 10pt font size (11 and 12 also possible), A4 paper (letterpaper for US letter) and two column layout (remove for one column)

\input{structure.tex} % Specifies the document structure and loads requires packages

%----------------------------------------------------------------------------------------
%	ARTICLE INFORMATION
%----------------------------------------------------------------------------------------

\title{Nat64 Support für IML} % The article title

\author{
	\authorstyle{Marco Romanutti\textsuperscript{1,2} und Benjamin Meyer\textsuperscript{1,2}} % Authors
	\newline\newline % Space before institutions
	\textsuperscript{1}\institution{Fachhochschule Nordwestschweiz FHNW, Brugg}\\ % Institution
	\textsuperscript{2}\texttt{Compilerbau, Klasse 7Ibb} % Module
}

\date{}

%----------------------------------------------------------------------------------------

\begin{document}

\maketitle % Print the title

\thispagestyle{firstpage} % Apply the page style for the first page (no headers and footers)

%----------------------------------------------------------------------------------------
%	ABSTRACT
%----------------------------------------------------------------------------------------

\lettrineabstract{Unter natürlichen Zahlen werden die positiven Zahlen und 0 verstanden. Die IML soll um einen neuen Datentyp nat64 erweitert werden. Der neue Datentyp soll solche Zahlen mit bis zu 64 Ziffern abbilden können. Es sollen die Operationen Addition, Subtraktion, Multiplikation, Division und Modulo  unterstützt werden.}
%----------------------------------------------------------------------------------------
%	ARTICLE CONTENTS
%----------------------------------------------------------------------------------------

\section{Einleitung}

Lorem ipsum dolor sit amet, consectetur adipisici elit, sed eiusmod tempor incidunt ut labore et dolore magna aliqua. Ut enim ad minim veniam, quis nostrud exercitation ullamco laboris nisi ut aliquid ex ea commodi consequat. Quis aute iure reprehenderit in voluptate velit esse cillum dolore eu fugiat nulla pariatur. Excepteur sint obcaecat cupiditat non proident, sunt in culpa qui officia deserunt mollit anim id est laborum.
\begin{align}
	A =
	\begin{bmatrix}
		0 & 0 & 0 & 0 & 0 & 0 & 0 & 0 & 0 & 0
	\end{bmatrix}
\end{align}
Duis autem vel eum iriure dolor in hendrerit in vulputate velit esse molestie consequat, vel illum dolore eu feugiat nulla facilisis.

\begin{lstlisting}
// Hello.java
import javax.swing.JApplet;
import java.awt.Graphics;

public class Hello extends JApplet {
    public void paintComponent(Graphics g) {
        g.drawString("Hello, world!", 65, 95);
    }
}
\end{lstlisting}

At vero eos et accusam et justo duo dolores et ea rebum. Stet clita kasd gubergren, no sea takimata sanctus est Lorem ipsum dolor sit amet. Lorem ipsum dolor sit amet, consetetur sadipscing elitr, sed diam nonumy eirmod tempor invidunt ut labore et dolore magna aliquyam erat, sed diam voluptua. At vero eos et accusam et justo duo dolores et ea rebum. Stet clita kasd gubergren, no sea takimata sanctus est Lorem ipsum dolor sit amet. Lorem ipsum dolor sit amet, consetetur sadipscing elitr, At accusam aliquyam diam diam dolore dolores duo eirmod eos erat, et nonumy sed tempor et et invidunt justo labore Stet clita ea et gubergren, kasd magna no rebum. sanctus sea sed takimata ut vero voluptua.

\begin{figure}[H]
    \includegraphics[width=\linewidth]{example_image.png} % Figure image
    \caption{This is a nice picture} % Figure caption
    \label{image_1} % Label for referencing with \ref{image_1}
\end{figure}

Consetetur sadipscing elitr, sed diam nonumy eirmod tempor invidunt ut labore et dolore magna aliquyam erat, sed diam voluptua. At vero eos et accusam et justo duo dolores et ea rebum. Stet clita kasd gubergren, no sea takimata sanctus est Lorem ipsum dolor sit amet. Lorem ipsum dolor sit amet, consetetur sadipscing elitr, sed diam nonumy eirmod tempor invidunt ut labore et dolore magna aliquyam erat, sed diam voluptua. At vero eos et accusam et justo duo dolores et ea rebum. Stet clita kasd gubergren, no sea takimata sanctus est Lorem ipsum dolor sit amet. Lorem ipsum dolor sit amet, consetetur sadipscing elitr, sed diam nonumy eirmod tempor invidunt ut labore et dolore magna aliquyam erat, sed diam voluptua. At vero eos et accusam et justo duo dolores et ea rebum. Stet clita kasd gubergren, no sea takimata sanctus est Lorem ipsum dolor sit amet.

Consetetur sadipscing elitr, sed diam nonumy eirmod tempor invidunt ut labore et dolore magna aliquyam erat, sed diam voluptua. At vero eos et accusam et justo duo dolores et ea rebum. Stet clita kasd gubergren, no sea takimata sanctus est Lorem ipsum dolor sit amet. Lorem ipsum dolor sit amet, consetetur sadipscing elitr, sed diam nonumy eirmod tempor invidunt ut labore et dolore magna aliquyam erat, sed diam voluptua. At vero eos et accusam et justo duo dolores et ea rebum. Stet clita kasd gubergren, no sea takimata sanctus est Lorem ipsum dolor sit amet. Lorem ipsum dolor sit amet, consetetur sadipscing elitr, sed diam nonumy eirmod tempor invidunt ut labore et dolore magna aliquyam erat, sed diam voluptua.

\begin{table}[h]
    \begin{tabularx}{\linewidth}{>{\parskip1ex}X@{\kern4\tabcolsep}>{\parskip1ex}X}
        \toprule
        \hfil\bfseries Pros
        &
        \hfil\bfseries Cons
        \\\cmidrule(r{3\tabcolsep}){1-1}\cmidrule(l{-\tabcolsep}){2-2}

        %% PROS, seperated by empty line or \par
        \lipsum[1] Anstelle der eigentlichen Daten werden nur die Hashes der Elemente gespeichert. Dadurch wird weniger Speicherkapazität benötigt.
        \par
        \lipsum[2] Das Hinzufügen und Prüfen von Elementen gehört zur Komplexitätsklasse \textit{O(k)}. Weil die verschiedenen Hashverfahren voneinander unabhängig sind, kann die Anwendung ebendieser parallelisiert werden.
        \par
        \lipsum[3] Durch die Anwendung von Hashfunktionen auf die einzelnen Werte kann der Nachteil von unregelmässig verteilten Daten verringert werden. In verteilten Datenbanken wird durch das Hashing-Verfahren eine gleichmässigere Verteilung der Daten auf den verschiedenen Datenbank-Nodes erreicht.

        &
        %% CONS, seperated by empty line or \par
        \lipsum[1] Ein Bloom-Filter kann erkennen, ob ein Wort \textit{nicht} im Filter vorhanden ist - ob ein Wort allerdings mit Sicherheit im Filter vorkommt kann nicht bestimmt werden.
        \par
        \lipsum[2] Der Bloom-Filter ist einfach umzusetzen, falls Wörter nur hinzugefügt werden. Die eingangs beschriebene Funktionsweise eignet sich allerdings nicht, falls Wörter auch entfernt oder zu einem späteren Zeitpunkt geändert werden müssen.
        \par
        \lipsum[3] Wird ein Bloom-Filter erstellt, so werden die damit verbundenen Daten nicht gespeichert. Dies führt zur Einschränkung, dass nicht auf die Daten zugegriffen werden kann, da nur deren Hash-Werte im Bloom-Filter gespeichert sind.

        \\\bottomrule
    \end{tabularx}
    \caption{Gegenüberstellung Vorteile und Nachteile}
\end{table}


%----------------------------------------------------------------------------------------
%	BIBLIOGRAPHY
%----------------------------------------------------------------------------------------

\begin{thebibliography}{9}
	\bibitem{wikipedia}
	Wikipedia: Natürliche Zahl,
	\url{https://de.wikipedia.org/wiki/Nat\%C3\%BCrliche_Zahl}

    \bibitem{wikipedia}
    Wikipedia: Natural numbers (engl.),
    \url{https://en.wikipedia.org/wiki/Natural_number}


\end{thebibliography}
%----------------------------------------------------------------------------------------

\end{document}
