%%%%%%%%%%%%%%%%%%%%%%%%%%%%%%%%%%%%%%%%%
% Nat64 support for IML
% Module cpib
%
% This template was downloaded from:
% http://www.LaTeXTemplates.com
%
% Authors:
% Marco Romanutti
% Benjamin Meyer
%
% License:
% CC BY-NC-SA 3.0 (http://creativecommons.org/licenses/by-nc-sa/3.0/)
%
%%%%%%%%%%%%%%%%%%%%%%%%%%%%%%%%%%%%%%%%%

%----------------------------------------------------------------------------------------
%	PACKAGES AND OTHER DOCUMENT CONFIGURATIONS
%----------------------------------------------------------------------------------------

\documentclass[10pt, a4paper, twocolumn]{article} % 10pt font size (11 and 12 also possible), A4 paper (letterpaper for US letter) and two column layout (remove for one column)

\input{structure.tex} % Specifies the document structure and loads requires packages

%----------------------------------------------------------------------------------------
%	ARTICLE INFORMATION
%----------------------------------------------------------------------------------------

\title{Nat64 und Casting Support für IML} % The article title

\author{
	\authorstyle{Marco Romanutti\textsuperscript{1,2} und Benjamin Meyer\textsuperscript{1,2}} % Authors
	\newline\newline % Space before institutions
	\textsuperscript{1}\institution{Fachhochschule Nordwestschweiz FHNW, Brugg}\\ % Institution
	\textsuperscript{2}\texttt{Zwischenbericht} % Module
}

\date{}

%----------------------------------------------------------------------------------------

\begin{document}

\maketitle % Print the title

\thispagestyle{firstpage} % Apply the page style for the first page (no headers and footers)

%----------------------------------------------------------------------------------------
%	ABSTRACT
%----------------------------------------------------------------------------------------

%\lettrineabstract{Im Modul Compilerbau soll eine Erweiterung für eine bestehende IML spezifiziert und implementiert werden. Unsere Erweiterung führt die Unterstützung eines Datentypen für natürliche Zahlen ein. In den nachfolgegenden Abschnitten werden die Erweiterungen an der IML-Spezifikation beschrieben und Implementationsentscheide begründet.}
\lettrineabstract{Im Modul Compilerbau wird eine Erweiterung für die bestehende IML spezifiziert und implementiert. Neu soll ein Datentyp für natürliche Zahlen unterstützt werden. Werte vom bestehenden Datentyp int64 können in den neuen Datentypen gecastet werden und umgekehrt.}

%----------------------------------------------------------------------------------------
%	ARTICLE CONTENTS
%----------------------------------------------------------------------------------------

\section{Compiler}
Der Compiler basiert auf der IML (V2) und ist in Java geschrieben.

\section{Erweiterung}
\subsection{Einleitung}
Unter natürlichen Zahlen werden die positiven Zahlen und 0 verstanden.
Die IML soll um einen neuen Datentyp \texttt{nat64} erweitert werden.
Der neue Datentyp soll solche Zahlen mit bis zu 64 Ziffern abbilden können.
Es sollen die Operationen Addition, Subtraktion, Multiplikation, Division und Modulo unterstützt werden.
Ausserdem soll ein explizites Casting zwischen dem bestehenden Datentyp \texttt{int64} und dem neuen Datentyp \texttt{nat64} möglich sein.

\subsection{Lexikalische Syntax}
Für den neuen Datentyp wird das Keyword \texttt{(TYPE, NAT64)} hinzugefügt.
Ebenfalls wird ein explizites Casting implementiert.
Casting ist nur von \texttt{(TYPE, INT64)} zu \texttt{(TYPE, NAT64)} und umgekehrt möglich.
Als Castingoperator wird die rechtekige Klammern (nachfolgend Brackets gennant) verwendet.
Innerhalb der Brakets befindet sich der Zieldatentyp (zum Beispiel \texttt{[int64]}).


% Listing mit neuen Elementen
\begin{lstlisting}[backgroundcolor = \color{lightgray},
xleftmargin = 0.05cm,
framexleftmargin = 0.05em]
    Datentyp:     \texttt{nat64}     \texttt{(TYPE, NAT64)}
    Brackets:     [ ]       LBRACKET, RBRACKET
\end{lstlisting}

\subsection{Grammatikalische Syntax}
Das nachfolgende Code-Listing zeigt, wie der neue Datentyp \texttt{nat64} eingesetzt werden kann.
\begin{lstlisting}
// Deklaration
var natIdent1 : nat64;
var natIdent2 : nat64;
var natIdent3 : nat64;

// Initialisierung
natIdent1 init := 50;
natIdent2 init := 10;
natIdent3 init := natIdent1 + natIdent2;

// Casting von int64 nach nat64
var intIdent1 : int;
intIdent1 init := 30
natIdent3 := [nat64] intIdent1;

call functionWithNatParam([nat64] intIdent1);

// Casting von nat64 nach int64
var intIdent2 : int;
intIdent2 init := [int64] natIdent3;

call functionWithIntParam([int64] natIdent3);
\end{lstlisting}
Falls zwei Datentypen nicht gecastet werden können, wird ein Kompilierungsfehler geworfen.
Folgendes Code-Listing zeigt ein solches Beispiel mit dem bestehenden Datentyp \texttt{bool}:
\begin{lstlisting}
// Deklaration
var boolIdent : bool;
boolIdent init := false;

var natIdent : nat64;
// Throws error:
natIdent init := [nat64] boolIdent
\end{lstlisting}
Unsere Erweiterung unterstützt keine impliziten Castings.
Weitere Code-Beispiele sind in Kapitel \ref{sec:prog} zu finden.

\subsection{Änderungen an der Grammatik}

Zusätzlich zu den bestehende Operatoren wurde ein neuer \texttt{castOpr} erstellt, welcher anstelle des Nichtterminal-Symbol \texttt{factor} verwendet werden kann.
% Neuer Operator
\begin{lstlisting}[backgroundcolor = \color{lightgray},
xleftmargin = 0.05cm,
framexleftmargin = 0.05em]
    castOpr :=  LBRACKET LITERAL RBRACKET
\end{lstlisting}
Das bestehende Nichtterminal-Symbol \texttt{factor} wird um diese neue Produktion ergänzt:
% Neue Produktionen
\begin{lstlisting}[backgroundcolor = \color{lightgray},
xleftmargin = 0.05cm,
framexleftmargin = 0.05em]
    factor :=   LITERAL
    | IDENT [INIT | exprList]
    | castOpr factor
    | monadicOpr factor
    | LPAREN expr RPAREN
\end{lstlisting}

\subsection{Kontext- und Typen-Einschränkungen}
Der Literal zwischen \texttt{LBRACKET} und \texttt{RBRACKET} muss vom Datentyp \texttt{int64} oder \texttt{nat64} sein.
Ein Casting zum Typ \texttt{bool} oder vom Typ \texttt{bool} zu \texttt{int64} resp. \texttt{nat64} führt zu einem Kompilierungsfehler.

Der neue Datentyp \texttt{nat64} unterstützt die bestehenden Operationen aus IML\footnote{Aktuell sind dies \begin{itemize} \item MULTOPR(*, divE, modE) \item ADDOPR(+, -) \item RELOPR(<, <=, >, >=, =, /=) \item BOOLOPR(/\textbackslash? \textbackslash/?)\end{itemize}}.
Sofern sich die einzelnen Operanden und auch das Resulat im Wertebereich ($\in \mathbb{N}$) befinden, entspricht das Verhalten vom Datentyp \texttt{nat64} jenem vom Datentyp \texttt{int64}.
Andernfalls wird folgendes Verhalten festgelegt:

\begin{itemize}
    \item \textbf{Werteüberlauf}: Bei einem überlauf wird ein Laufzeitfehler geworfen % TODO --> soll etwas anderes passieren?
    \item \textbf{Negative Werte}: Wird ein Laufzeitfehler geworfen. Bei einem Casting wird der negative Wert mit (-1) multipliziert % TODO --> Soll etwas anderes passieren?
    \item \textbf{Rest bei Division}: Wird analog \texttt{int64} behandelt und Nachkommastellen werden abgeschnitten.
\end{itemize}

Tabelle \ref{tab:Casting} zeigt die unterstützen Typumwandlungen der verschiedenen Datentypen.
Typumwandlungen, welche zu potentiellem Informationsverlust führen, sind mit mit \texttt{*} gekennzeichnet.
% TODO Handhabung von diesen Umwandungen bei Wertüberlauf, negativen Werten definieren, s. oben --> soll etwas anderes passieren?
\begin{table}[h]
    \tiny
    \centering
    \caption{Casting zwischen Datentypen}
    \label{tab:Casting}
    \resizebox{\columnwidth}{!}{%
    \begin{tabular}{rlll}
        \hline %TODO wollen wir dieses Verhalten? Zudem wird die Fussnote nicht richtig dargestellt
        Quell- \textbackslash \ Zieldatentyp & \texttt{int64} & \texttt{nat64} \\ \hline
        \texttt{int64}                   & \cmark        & \cmark * \footnote{value >= 0 -> normale konvertierung, value < 0 -> value * (-1)}       & \xmark     \\
        \texttt{nat64}                   & \cmark        & \cmark         & \xmark     \\
    \end{tabular}%
    }
\end{table}


\section{Vergleich mit anderen Programmiersprachen}
\subsection{Java}
In Java werden Wertebereichsunterschreibungen und -überschreibungen für Initialisierung und fortlaufende Berechnungen unterschiedliche gehandhabt:
Bei der Initialisierung wird beispielsweise beim Datentyp \texttt{int} geprüft, ob die Werte innerhalb der Wertebereiche liegen. Falls dies nicht der Fall ist, wird ein Error geworfen.
Falls bei fortlaufenden Berechnungen Wertebereiche unter- resp. überschritten werden, wird einfach auf dem Zahlenkreis weitergegangen und der entsprechende Wert verwendet.
Dies kann dazu führen, dass mit \glqq falschen\grqq \ Werten gerechnet wird, ohne dass der Entwickler dies bemerkt.

\section{Beispielprogramme}
\label{sec:prog}
Operation:
\begin{lstlisting}
program progAddition
global
    var x:nat64;
    var y:nat64;
    var r:nat64;
    var b:bool
do
    x init := 4;
    y init := 3;
    r init := x + y;
    b init := r = 7;

    debugout r;
    debugout b
endprogram
\end{lstlisting}
Casting:
\begin{lstlisting}
program progCasting
global
    var x:nat64;
    var y:int64;
    var r:nat64;
    var b:bool
do
    x init := 4;
    y init := 3;
    r init := x + [nat64] y;
    b init := r = 7;

    debugout r;
    debugout b
endprogram
\end{lstlisting}

%----------------------------------------------------------------------------------------
%	BIBLIOGRAPHY
%----------------------------------------------------------------------------------------

\begin{thebibliography}{9}
	\bibitem{wikipedia}
	Wikipedia: Natürliche Zahl,
	\url{https://de.wikipedia.org/wiki/Nat\%C3\%BCrliche_Zahl}

    \bibitem{wikipedia}
    Wikipedia: Natural numbers (engl.),
    \url{https://en.wikipedia.org/wiki/Natural_number}


\end{thebibliography}
%----------------------------------------------------------------------------------------

\end{document}
