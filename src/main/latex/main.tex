%%%%%%%%%%%%%%%%%%%%%%%%%%%%%%%%%%%%%%%%%
% Nat64 support for IML
% Module cpib
%
% This template was downloaded from:
% http://www.LaTeXTemplates.com
%
% Authors:
% Marco Romanutti
% Benjamin Meyer
%
% License:
% CC BY-NC-SA 3.0 (http://creativecommons.org/licenses/by-nc-sa/3.0/)
%
%%%%%%%%%%%%%%%%%%%%%%%%%%%%%%%%%%%%%%%%%

%----------------------------------------------------------------------------------------
%	PACKAGES AND OTHER DOCUMENT CONFIGURATIONS
%----------------------------------------------------------------------------------------

\documentclass[10pt, a4paper, twocolumn]{article} % 10pt font size (11 and 12 also possible), A4 paper (letterpaper for US letter) and two column layout (remove for one column)

\input{structure.tex} % Specifies the document structure and loads requires packages

%----------------------------------------------------------------------------------------
%	ARTICLE INFORMATION
%----------------------------------------------------------------------------------------

\title{Nat64 Support für IML} % The article title

\author{
	\authorstyle{Marco Romanutti\textsuperscript{1,2} und Benjamin Meyer\textsuperscript{1,2}} % Authors
	\newline\newline % Space before institutions
	\textsuperscript{1}\institution{Fachhochschule Nordwestschweiz FHNW, Brugg}\\ % Institution
	\textsuperscript{2}\texttt{Zwischenbericht Compilerbau} % Module
}

\date{}

%----------------------------------------------------------------------------------------

\begin{document}

\maketitle % Print the title

\thispagestyle{firstpage} % Apply the page style for the first page (no headers and footers)

%----------------------------------------------------------------------------------------
%	ABSTRACT
%----------------------------------------------------------------------------------------

\lettrineabstract{Im Modul Compilerbau soll eine Erweiterung für eine bestehende IML spezifiziert und implementiert werden. Unsere Erweiterung führt die Unterstützung eines Datentypen für natürliche Zahlen ein. In den nachfolgegenden Abschnitten werden die Erweiterungen an der IML-Spezifikation beschrieben und Implementationsentscheide begründet.}
%----------------------------------------------------------------------------------------
%	ARTICLE CONTENTS
%----------------------------------------------------------------------------------------

\section{Einleitung}

Unter natürlichen Zahlen werden die positiven Zahlen und 0 verstanden. Die IML soll um einen neuen Datentyp nat64 erweitert werden. Der neue Datentyp soll solche Zahlen mit bis zu 64 Ziffern abbilden können. Es sollen die Operationen Addition, Subtraktion, Multiplikation, Division und Modulo  unterstützt werden. Ausserdem soll ein explizites Casting zwischen dem bestehenden Datentyp int64 und dem neuen Datentyp nat64 möglich sein.

\section{Lexikalische Syntax}
Für den neuen Datentyp wird ein neues Keyword \texttt{(TYPE, NAT64)} eingefügt.
Zusätzlich werden für das Casting rechteckige Klammern eingeführt.
Die rechteckigen Klammern werden nachfolgend Brackets genannt.
Mithilfe der Brackets kann der Ziel-Datentyp bestimmt werden.
Dieser wird dabei zwischen einer öffnenden Klammer \texttt{(LBRACKET)} und einer schliessenden Klammer \texttt{(RBRACKET)} angegeben.

% Tabelle für neue Elemente
\begin{table}[h]
\centering
\small
\begin{tabular}{lll}
Neuer Datentyp: & nat64   & (TYPE, NAT64)      \\
Neue Brackets: & [ ] & LBRACKET, RBRACKET
\end{tabular}
\end{table}

\section{Grammatikalische Syntax}
Das nachfolgende Code-Listing zeigt, wie der neue Datentyp eingesetzt werden kann.
\begin{lstlisting}
    // Deklaration
    var natIdent1 : nat64;
    var natIdent2 : nat64;
    var natIdent3 : nat64;

    // Initialisierung
    natIdent1 init := 50;
    natIdent2 init := 10;
    natIdent3 init := natIdent1 + natIdent2;

    // Casting
    // int64 -> nat64
    var intIdent1 : int;
    intIdent1 init := 30
    natIdent3 := [nat64] intIdent1;

    call functionWithNatParam([nat64] intIdent1);

    // nat64 -> int64
    var intIdent2 : int;
    intIdent2 init := [int64] natIdent3;

    call functionWithIntParam([int64] natIdent3);
\end{lstlisting}

Falls zwei Datentypen nicht gecastet werden können, muss ein Grammatik-Error geworften werden.
\begin{lstlisting}
    // Deklaration
    var boolIdent : bool;
    boolIdent init := false;

    var natIdent : nat64;
    natIdent init := [nat64] boolIdent // Throws Error!
\end{lstlisting}

\section{Änderungen an der Grammatik}

Zusätzlich zu den bestehende Operatoren wurde ein neuer \texttt{castOpr} erstellt, welcher in Zusammenhang mit dem Nichtterminal-Symbol \texttt{factor} verwendet werden kann.

% Tabelle für neue Elemente
\begin{table}[h]
    \centering
    \small
    \begin{tabular}{l}
        castOpr := LBRACKET LITERAL RBRACKET
    \end{tabular}
\end{table}

Der bestehende \texttt{factor} wird um den neuen Operator ergänzt:

\begin{table}[h]
    \centering
    \small
    \begin{tabular}{ll}
        factor :=   & LITERAL \\
                    & | IDENT [INIT | exprList] \\
                    & | castOpr factor \\
                    & | monadicOpr factor \\
                    & | LPAREN expr RPAREN
    \end{tabular}
\end{table}

\section{Kontext- und Typen-Einschränkungen}

\section{Vergleich mit anderen Programmiersprachen}


%----------------------------------------------------------------------------------------
%	BIBLIOGRAPHY
%----------------------------------------------------------------------------------------

\begin{thebibliography}{9}
	\bibitem{wikipedia}
	Wikipedia: Natürliche Zahl,
	\url{https://de.wikipedia.org/wiki/Nat\%C3\%BCrliche_Zahl}

    \bibitem{wikipedia}
    Wikipedia: Natural numbers (engl.),
    \url{https://en.wikipedia.org/wiki/Natural_number}


\end{thebibliography}
%----------------------------------------------------------------------------------------

\end{document}
